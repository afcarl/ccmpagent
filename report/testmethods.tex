\section{Test Methodologies}
To test the behaviour and stability of the CCMP Agent framework and the
associated decision tree and trust network implementations several different test
strategies were adopted.

\subsection{CCMP Agent Framework Testing}
The CCMP Agent Framework was first tested by implementing the
Simple Agent example provided in the ART testbed.  A SimpleDT and
SimpleTrust implementation of the associated interface classes were created.
The SimpleDT class implemented the decision methods using the same if conditions
as the example agent.  The SimpleTrust class merely provided trust values of
1.0, the same as the example agent.  A SimpleCCMPAgent was then created that
instantiates the SimpleDT and SimpleTrust classes.  The behaviour of the
SimpleCCMPAgent was then compared to the behaviour of the example SimpleAgent
to ensure the CCMP Agent Framework implemented the ART methods correctly.

Logging to a file was also implemented in the CCMPAgent class.  The CCMPAgent
class will print information on the messages received, decisions taken and the
messages generated.  It also logs any negative actions taken by other agents.
With the use of logging, the complete behaviour of the SimpleCCMPAgent and
associated  could be examined to ensure the expected behaviour was being
performed.

\subsection{Weka Decision Tree Testing}
The Weka Decision Tree implementation was tested in two ways.  The first
method was to create classes that would test the parsing of the XML config
file, which contains the Weka data, and to test the creation and execution of
the Weka decision tree parsing.

The second requirement for testing of the Weka Decision Tree was to ensure
that the behaviour of the Weka decision trees generated from the Weka data
provided in the XML config file was as expected.  To speed up the process of
verifying the Weka decision trees a class was created to visualize the Weka
decision trees.  The XML config file is parsed, Weka decision tree created and
then displayed using the Weka visualization tools.

\subsection{Bayes Trust Framework Testing}
The Bayesian trust component of the project was quality-checked by unit testing.
Using the JUnit testing framework, we built unit tests for all 13 of our concrete
classes in the Bayesian trust library. These tests include positive tests, which
check for correct behaviour, and negative tests, which ensure correct error
handling under exceptional conditions. The tests were also useful as regression
tests, i.e., to ensure that previously fixed bugs are not reintroduced as a
result of new code. In total, we have 58 test cases which verify the Bayesian
trust library.

