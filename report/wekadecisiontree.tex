\section{Weka Decision Tree}
The decision was made early that developing an advanced tree classifier from scratch would
be unnecessary as the Weka 3\cite{weka} open source library already provided an implementation that meet
all the requirements.  These requirements including being able to classify data sets and 
then being able to query the resulting tree to determine the categorical action based on the set
of noncategorical test.  Also, there was a need to view the resulting tree for visual inspection.

\subsection{Interfacing with Weka}
In order to increase the reusability of the decision tree library, classes were created that 
reproduce the components of the Weka Attribute-Relation File Format (ARFF) instead of merely
storing the contents of an ARFF file.  These classes are of this decision tree library, dtlib, 
are described in the following table.

\begin{center}
\begin{tabular}{|l|l|}
\hline
DTAttribute &
Contains two strings representing the name of the attribute and its type.
The type can be either 'numeric' or some enumerated set of the form '\{enum1,enum2,etc\}'.\\
\hline
DTWekaARFF &
Contains a string for the name of the tree, a vector of the noncategorical attributes and the 
categorical attribute as the last element, and a String array representing the data that the 
decision tree will be built from.\\
\hline
DTLearning &
Contains a DTWekaARFF object and a J48 tree object.  The tree object is retrieved by converting
the DTWekaARFF element to the ARFF format and passing it to Weka's BuildClassifier.  The
DTClassify method will return the proper action to be taken based on passed test\\
\hline
DTLearningCollection &
Extends Vector to only contain DTLearning objects.
\hline
\end{tabular}
\end{center}

The decision trees were stored in configuration XML allowing the behaviour of an agent to be change without
recompiling the entire project.  This flexibility was achieved by structuring the classes so that 
they could be built using the Digester XML parser.





